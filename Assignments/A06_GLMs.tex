% Options for packages loaded elsewhere
\PassOptionsToPackage{unicode}{hyperref}
\PassOptionsToPackage{hyphens}{url}
%
\documentclass[
]{article}
\title{Assignment 6: GLMs (Linear Regressios, ANOVA, \& t-tests)}
\author{Curtis Cha}
\date{}

\usepackage{amsmath,amssymb}
\usepackage{lmodern}
\usepackage{iftex}
\ifPDFTeX
  \usepackage[T1]{fontenc}
  \usepackage[utf8]{inputenc}
  \usepackage{textcomp} % provide euro and other symbols
\else % if luatex or xetex
  \usepackage{unicode-math}
  \defaultfontfeatures{Scale=MatchLowercase}
  \defaultfontfeatures[\rmfamily]{Ligatures=TeX,Scale=1}
\fi
% Use upquote if available, for straight quotes in verbatim environments
\IfFileExists{upquote.sty}{\usepackage{upquote}}{}
\IfFileExists{microtype.sty}{% use microtype if available
  \usepackage[]{microtype}
  \UseMicrotypeSet[protrusion]{basicmath} % disable protrusion for tt fonts
}{}
\makeatletter
\@ifundefined{KOMAClassName}{% if non-KOMA class
  \IfFileExists{parskip.sty}{%
    \usepackage{parskip}
  }{% else
    \setlength{\parindent}{0pt}
    \setlength{\parskip}{6pt plus 2pt minus 1pt}}
}{% if KOMA class
  \KOMAoptions{parskip=half}}
\makeatother
\usepackage{xcolor}
\IfFileExists{xurl.sty}{\usepackage{xurl}}{} % add URL line breaks if available
\IfFileExists{bookmark.sty}{\usepackage{bookmark}}{\usepackage{hyperref}}
\hypersetup{
  pdftitle={Assignment 6: GLMs (Linear Regressios, ANOVA, \& t-tests)},
  pdfauthor={Curtis Cha},
  hidelinks,
  pdfcreator={LaTeX via pandoc}}
\urlstyle{same} % disable monospaced font for URLs
\usepackage[margin=2.54cm]{geometry}
\usepackage{color}
\usepackage{fancyvrb}
\newcommand{\VerbBar}{|}
\newcommand{\VERB}{\Verb[commandchars=\\\{\}]}
\DefineVerbatimEnvironment{Highlighting}{Verbatim}{commandchars=\\\{\}}
% Add ',fontsize=\small' for more characters per line
\usepackage{framed}
\definecolor{shadecolor}{RGB}{248,248,248}
\newenvironment{Shaded}{\begin{snugshade}}{\end{snugshade}}
\newcommand{\AlertTok}[1]{\textcolor[rgb]{0.94,0.16,0.16}{#1}}
\newcommand{\AnnotationTok}[1]{\textcolor[rgb]{0.56,0.35,0.01}{\textbf{\textit{#1}}}}
\newcommand{\AttributeTok}[1]{\textcolor[rgb]{0.77,0.63,0.00}{#1}}
\newcommand{\BaseNTok}[1]{\textcolor[rgb]{0.00,0.00,0.81}{#1}}
\newcommand{\BuiltInTok}[1]{#1}
\newcommand{\CharTok}[1]{\textcolor[rgb]{0.31,0.60,0.02}{#1}}
\newcommand{\CommentTok}[1]{\textcolor[rgb]{0.56,0.35,0.01}{\textit{#1}}}
\newcommand{\CommentVarTok}[1]{\textcolor[rgb]{0.56,0.35,0.01}{\textbf{\textit{#1}}}}
\newcommand{\ConstantTok}[1]{\textcolor[rgb]{0.00,0.00,0.00}{#1}}
\newcommand{\ControlFlowTok}[1]{\textcolor[rgb]{0.13,0.29,0.53}{\textbf{#1}}}
\newcommand{\DataTypeTok}[1]{\textcolor[rgb]{0.13,0.29,0.53}{#1}}
\newcommand{\DecValTok}[1]{\textcolor[rgb]{0.00,0.00,0.81}{#1}}
\newcommand{\DocumentationTok}[1]{\textcolor[rgb]{0.56,0.35,0.01}{\textbf{\textit{#1}}}}
\newcommand{\ErrorTok}[1]{\textcolor[rgb]{0.64,0.00,0.00}{\textbf{#1}}}
\newcommand{\ExtensionTok}[1]{#1}
\newcommand{\FloatTok}[1]{\textcolor[rgb]{0.00,0.00,0.81}{#1}}
\newcommand{\FunctionTok}[1]{\textcolor[rgb]{0.00,0.00,0.00}{#1}}
\newcommand{\ImportTok}[1]{#1}
\newcommand{\InformationTok}[1]{\textcolor[rgb]{0.56,0.35,0.01}{\textbf{\textit{#1}}}}
\newcommand{\KeywordTok}[1]{\textcolor[rgb]{0.13,0.29,0.53}{\textbf{#1}}}
\newcommand{\NormalTok}[1]{#1}
\newcommand{\OperatorTok}[1]{\textcolor[rgb]{0.81,0.36,0.00}{\textbf{#1}}}
\newcommand{\OtherTok}[1]{\textcolor[rgb]{0.56,0.35,0.01}{#1}}
\newcommand{\PreprocessorTok}[1]{\textcolor[rgb]{0.56,0.35,0.01}{\textit{#1}}}
\newcommand{\RegionMarkerTok}[1]{#1}
\newcommand{\SpecialCharTok}[1]{\textcolor[rgb]{0.00,0.00,0.00}{#1}}
\newcommand{\SpecialStringTok}[1]{\textcolor[rgb]{0.31,0.60,0.02}{#1}}
\newcommand{\StringTok}[1]{\textcolor[rgb]{0.31,0.60,0.02}{#1}}
\newcommand{\VariableTok}[1]{\textcolor[rgb]{0.00,0.00,0.00}{#1}}
\newcommand{\VerbatimStringTok}[1]{\textcolor[rgb]{0.31,0.60,0.02}{#1}}
\newcommand{\WarningTok}[1]{\textcolor[rgb]{0.56,0.35,0.01}{\textbf{\textit{#1}}}}
\usepackage{graphicx}
\makeatletter
\def\maxwidth{\ifdim\Gin@nat@width>\linewidth\linewidth\else\Gin@nat@width\fi}
\def\maxheight{\ifdim\Gin@nat@height>\textheight\textheight\else\Gin@nat@height\fi}
\makeatother
% Scale images if necessary, so that they will not overflow the page
% margins by default, and it is still possible to overwrite the defaults
% using explicit options in \includegraphics[width, height, ...]{}
\setkeys{Gin}{width=\maxwidth,height=\maxheight,keepaspectratio}
% Set default figure placement to htbp
\makeatletter
\def\fps@figure{htbp}
\makeatother
\setlength{\emergencystretch}{3em} % prevent overfull lines
\providecommand{\tightlist}{%
  \setlength{\itemsep}{0pt}\setlength{\parskip}{0pt}}
\setcounter{secnumdepth}{-\maxdimen} % remove section numbering
\ifLuaTeX
  \usepackage{selnolig}  % disable illegal ligatures
\fi

\begin{document}
\maketitle

\hypertarget{overview}{%
\subsection{OVERVIEW}\label{overview}}

This exercise accompanies the lessons in Environmental Data Analytics on
generalized linear models.

\hypertarget{directions}{%
\subsection{Directions}\label{directions}}

\begin{enumerate}
\def\labelenumi{\arabic{enumi}.}
\tightlist
\item
  Change ``Student Name'' on line 3 (above) with your name.
\item
  Work through the steps, \textbf{creating code and output} that fulfill
  each instruction.
\item
  Be sure to \textbf{answer the questions} in this assignment document.
\item
  When you have completed the assignment, \textbf{Knit} the text and
  code into a single PDF file.
\item
  After Knitting, submit the completed exercise (PDF file) to the
  dropbox in Sakai. Add your last name into the file name (e.g.,
  ``Fay\_A06\_GLMs.Rmd'') prior to submission.
\end{enumerate}

The completed exercise is due on Monday, February 28 at 7:00 pm.

\hypertarget{set-up-your-session}{%
\subsection{Set up your session}\label{set-up-your-session}}

\begin{enumerate}
\def\labelenumi{\arabic{enumi}.}
\item
  Set up your session. Check your working directory. Load the tidyverse,
  agricolae and other needed packages. Import the \emph{raw} NTL-LTER
  raw data file for chemistry/physics
  (\texttt{NTL-LTER\_Lake\_ChemistryPhysics\_Raw.csv}). Set date columns
  to date objects.
\item
  Build a ggplot theme and set it as your default theme.
\end{enumerate}

\begin{Shaded}
\begin{Highlighting}[]
\CommentTok{\#1}
\FunctionTok{library}\NormalTok{(tidyverse)}
\end{Highlighting}
\end{Shaded}

\begin{verbatim}
## -- Attaching packages --------------------------------------- tidyverse 1.3.1 --
\end{verbatim}

\begin{verbatim}
## v ggplot2 3.3.5     v purrr   0.3.4
## v tibble  3.1.6     v dplyr   1.0.7
## v tidyr   1.1.3     v stringr 1.4.0
## v readr   2.0.1     v forcats 0.5.1
\end{verbatim}

\begin{verbatim}
## -- Conflicts ------------------------------------------ tidyverse_conflicts() --
## x dplyr::filter() masks stats::filter()
## x dplyr::lag()    masks stats::lag()
\end{verbatim}

\begin{Shaded}
\begin{Highlighting}[]
\FunctionTok{library}\NormalTok{(lubridate)}
\end{Highlighting}
\end{Shaded}

\begin{verbatim}
## 
## Attaching package: 'lubridate'
\end{verbatim}

\begin{verbatim}
## The following objects are masked from 'package:base':
## 
##     date, intersect, setdiff, union
\end{verbatim}

\begin{Shaded}
\begin{Highlighting}[]
\FunctionTok{library}\NormalTok{(ggplot2)}

\NormalTok{ntl }\OtherTok{\textless{}{-}} \FunctionTok{read.csv}\NormalTok{(}
  \StringTok{\textquotesingle{}../Data/Raw/NTL{-}LTER\_Lake\_ChemistryPhysics\_Raw.csv\textquotesingle{}}\NormalTok{, }
  \AttributeTok{header =} \ConstantTok{TRUE}\NormalTok{)}

\NormalTok{ntl}\SpecialCharTok{$}\NormalTok{sampledate }\OtherTok{\textless{}{-}} \FunctionTok{mdy}\NormalTok{(ntl}\SpecialCharTok{$}\NormalTok{sampledate)}

\CommentTok{\#2}
\NormalTok{mytheme }\OtherTok{\textless{}{-}} \FunctionTok{theme\_classic}\NormalTok{(}\AttributeTok{base\_size =} \DecValTok{12}\NormalTok{) }\SpecialCharTok{+} 
  \FunctionTok{theme}\NormalTok{(}\AttributeTok{axis.text =} \FunctionTok{element\_text}\NormalTok{(}\AttributeTok{color =} \StringTok{"black"}\NormalTok{), }\AttributeTok{legend.position =} \StringTok{"right"}\NormalTok{)}
\end{Highlighting}
\end{Shaded}

\hypertarget{simple-regression}{%
\subsection{Simple regression}\label{simple-regression}}

Our first research question is: Does mean lake temperature recorded
during July change with depth across all lakes?

\begin{enumerate}
\def\labelenumi{\arabic{enumi}.}
\setcounter{enumi}{2}
\item
  State the null and alternative hypotheses for this question:
  \textgreater{} Answer: H0: Lake depth level has no significant effect
  in determining mean of recorded lake temperatures for the month of
  July. Ha: Lake depth level has a significant effect in determining
  mean of recorded lake temperatures for the month of July.
\item
  Wrangle your NTL-LTER dataset with a pipe function so that the records
  meet the following criteria:
\end{enumerate}

\begin{itemize}
\tightlist
\item
  Only dates in July.
\item
  Only the columns: \texttt{lakename}, \texttt{year4}, \texttt{daynum},
  \texttt{depth}, \texttt{temperature\_C}
\item
  Only complete cases (i.e., remove NAs)
\end{itemize}

\begin{enumerate}
\def\labelenumi{\arabic{enumi}.}
\setcounter{enumi}{4}
\tightlist
\item
  Visualize the relationship among the two continuous variables with a
  scatter plot of temperature by depth. Add a smoothed line showing the
  linear model, and limit temperature values from 0 to 35 °C. Make this
  plot look pretty and easy to read.
\end{enumerate}

\begin{Shaded}
\begin{Highlighting}[]
\CommentTok{\#4}
\NormalTok{ntl\_july }\OtherTok{\textless{}{-}}\NormalTok{ ntl }\SpecialCharTok{\%\textgreater{}\%} 
  \FunctionTok{mutate}\NormalTok{(}\AttributeTok{month =} \FunctionTok{month}\NormalTok{(ntl}\SpecialCharTok{$}\NormalTok{sampledate)) }\SpecialCharTok{\%\textgreater{}\%}
  \FunctionTok{filter}\NormalTok{(month }\SpecialCharTok{==} \DecValTok{7}\NormalTok{, }\SpecialCharTok{!}\FunctionTok{is.na}\NormalTok{(ntl}\SpecialCharTok{$}\NormalTok{temperature\_C)) }\SpecialCharTok{\%\textgreater{}\%}
  \FunctionTok{select}\NormalTok{(}\StringTok{"lakename"}\NormalTok{, }\StringTok{"year4"}\NormalTok{, }\StringTok{"daynum"}\NormalTok{, }\StringTok{"depth"}\NormalTok{, }\StringTok{"temperature\_C"}\NormalTok{)}

\CommentTok{\#5}
\FunctionTok{ggplot}\NormalTok{(}\AttributeTok{data =}\NormalTok{ ntl\_july, }\FunctionTok{aes}\NormalTok{(}\AttributeTok{x =}\NormalTok{ depth, }\AttributeTok{y =}\NormalTok{ temperature\_C)) }\SpecialCharTok{+}
  \FunctionTok{geom\_point}\NormalTok{() }\SpecialCharTok{+} \FunctionTok{geom\_smooth}\NormalTok{(}\AttributeTok{method =} \StringTok{"lm"}\NormalTok{) }\SpecialCharTok{+} \FunctionTok{ylim}\NormalTok{(}\FunctionTok{c}\NormalTok{(}\DecValTok{0}\NormalTok{,}\DecValTok{35}\NormalTok{)) }\SpecialCharTok{+} 
  \FunctionTok{xlab}\NormalTok{(}\StringTok{"Depth(m)"}\NormalTok{) }\SpecialCharTok{+} \FunctionTok{ylab}\NormalTok{(}\StringTok{"Temperature(C)"}\NormalTok{) }\SpecialCharTok{+} 
  \FunctionTok{ggtitle}\NormalTok{(}\StringTok{"Mean Recorded Temperature of Lakes (July)}\SpecialCharTok{\textbackslash{}n}\StringTok{ vs. Depth Level"}\NormalTok{)}
\end{Highlighting}
\end{Shaded}

\begin{verbatim}
## `geom_smooth()` using formula 'y ~ x'
\end{verbatim}

\begin{verbatim}
## Warning: Removed 24 rows containing missing values (geom_smooth).
\end{verbatim}

\includegraphics{A06_GLMs_files/figure-latex/scatterplot-1.pdf}

\begin{enumerate}
\def\labelenumi{\arabic{enumi}.}
\setcounter{enumi}{5}
\tightlist
\item
  Interpret the figure. What does it suggest with regards to the
  response of temperature to depth? Do the distribution of points
  suggest about anything about the linearity of this trend?
\end{enumerate}

\begin{quote}
Answer: The scatter plot suggests that depth has a negative relationship
with mean recorded temperature. The distribution of points suggest there
may not be a linear relationship between the two variables sa there is
high variance in mean temperatures at lower depts (0-5m). However at
lower depths (10-15m), there is consistent low mean temperatures. It's
possible that there true relationship between depth and temperature is
not linear.
\end{quote}

\begin{enumerate}
\def\labelenumi{\arabic{enumi}.}
\setcounter{enumi}{6}
\tightlist
\item
  Perform a linear regression to test the relationship and display the
  results
\end{enumerate}

\begin{Shaded}
\begin{Highlighting}[]
\CommentTok{\#7}

\NormalTok{lm0 }\OtherTok{\textless{}{-}} \FunctionTok{lm}\NormalTok{(}\AttributeTok{data =}\NormalTok{ ntl\_july, temperature\_C }\SpecialCharTok{\textasciitilde{}}\NormalTok{ depth)}
\NormalTok{sum\_0 }\OtherTok{\textless{}{-}} \FunctionTok{summary}\NormalTok{(lm0)}
\FunctionTok{print}\NormalTok{(sum\_0)}
\end{Highlighting}
\end{Shaded}

\begin{verbatim}
## 
## Call:
## lm(formula = temperature_C ~ depth, data = ntl_july)
## 
## Residuals:
##     Min      1Q  Median      3Q     Max 
## -9.5173 -3.0192  0.0633  2.9365 13.5834 
## 
## Coefficients:
##             Estimate Std. Error t value Pr(>|t|)    
## (Intercept) 21.95597    0.06792   323.3   <2e-16 ***
## depth       -1.94621    0.01174  -165.8   <2e-16 ***
## ---
## Signif. codes:  0 '***' 0.001 '**' 0.01 '*' 0.05 '.' 0.1 ' ' 1
## 
## Residual standard error: 3.835 on 9726 degrees of freedom
## Multiple R-squared:  0.7387, Adjusted R-squared:  0.7387 
## F-statistic: 2.75e+04 on 1 and 9726 DF,  p-value: < 2.2e-16
\end{verbatim}

\begin{enumerate}
\def\labelenumi{\arabic{enumi}.}
\setcounter{enumi}{7}
\tightlist
\item
  Interpret your model results in words. Include how much of the
  variability in temperature is explained by changes in depth, the
  degrees of freedom on which this finding is based, and the statistical
  significance of the result. Also mention how much temperature is
  predicted to change for every 1m change in depth.
\end{enumerate}

\begin{quote}
Answer: According to the model, for every one unit increase in depth,
there is a 1.95621 decrease in recorded mean temperature. Given the null
hypothesis is true, the probability of observing this relationship
between temperature and depth is significantly low (less than 0.05). Due
to this low p-value, I reject the null hypothesis that depth has no
significant effect on mean recorded lake temperature in July. The linear
model produce explains around 73.87\% of the variability in mean
recorded temperature.
\end{quote}

\begin{center}\rule{0.5\linewidth}{0.5pt}\end{center}

\hypertarget{multiple-regression}{%
\subsection{Multiple regression}\label{multiple-regression}}

Let's tackle a similar question from a different approach. Here, we want
to explore what might the best set of predictors for lake temperature in
July across the monitoring period at the North Temperate Lakes LTER.

\begin{enumerate}
\def\labelenumi{\arabic{enumi}.}
\setcounter{enumi}{8}
\item
  Run an AIC to determine what set of explanatory variables (year4,
  daynum, depth) is best suited to predict temperature.
\item
  Run a multiple regression on the recommended set of variables.
\end{enumerate}

\begin{Shaded}
\begin{Highlighting}[]
\CommentTok{\#9}

\NormalTok{lm1 }\OtherTok{\textless{}{-}} \FunctionTok{lm}\NormalTok{(}\AttributeTok{data =}\NormalTok{ ntl\_july, temperature\_C }\SpecialCharTok{\textasciitilde{}}\NormalTok{ year4 }\SpecialCharTok{+}\NormalTok{ daynum }\SpecialCharTok{+}\NormalTok{ depth)}

\NormalTok{lm\_final }\OtherTok{\textless{}{-}} \FunctionTok{step}\NormalTok{(lm1)}
\end{Highlighting}
\end{Shaded}

\begin{verbatim}
## Start:  AIC=26065.53
## temperature_C ~ year4 + daynum + depth
## 
##          Df Sum of Sq    RSS   AIC
## <none>                141687 26066
## - year4   1       101 141788 26070
## - daynum  1      1237 142924 26148
## - depth   1    404475 546161 39189
\end{verbatim}

\begin{Shaded}
\begin{Highlighting}[]
\FunctionTok{print}\NormalTok{(lm\_final}\SpecialCharTok{$}\NormalTok{coefficients) }\CommentTok{\#full model is the best model}
\end{Highlighting}
\end{Shaded}

\begin{verbatim}
## (Intercept)       year4      daynum       depth 
## -8.57556399  0.01134486  0.03978013 -1.94643682
\end{verbatim}

\begin{Shaded}
\begin{Highlighting}[]
\CommentTok{\#10}
\NormalTok{lm1 }\OtherTok{\textless{}{-}} \FunctionTok{lm}\NormalTok{(}\AttributeTok{data =}\NormalTok{ ntl\_july, temperature\_C }\SpecialCharTok{\textasciitilde{}}\NormalTok{ year4 }\SpecialCharTok{+}\NormalTok{ daynum }\SpecialCharTok{+}\NormalTok{ depth)}
\FunctionTok{print}\NormalTok{(}\FunctionTok{summary}\NormalTok{(lm1))}
\end{Highlighting}
\end{Shaded}

\begin{verbatim}
## 
## Call:
## lm(formula = temperature_C ~ year4 + daynum + depth, data = ntl_july)
## 
## Residuals:
##     Min      1Q  Median      3Q     Max 
## -9.6536 -3.0000  0.0902  2.9658 13.6123 
## 
## Coefficients:
##              Estimate Std. Error  t value Pr(>|t|)    
## (Intercept) -8.575564   8.630715   -0.994  0.32044    
## year4        0.011345   0.004299    2.639  0.00833 ** 
## daynum       0.039780   0.004317    9.215  < 2e-16 ***
## depth       -1.946437   0.011683 -166.611  < 2e-16 ***
## ---
## Signif. codes:  0 '***' 0.001 '**' 0.01 '*' 0.05 '.' 0.1 ' ' 1
## 
## Residual standard error: 3.817 on 9724 degrees of freedom
## Multiple R-squared:  0.7412, Adjusted R-squared:  0.7411 
## F-statistic:  9283 on 3 and 9724 DF,  p-value: < 2.2e-16
\end{verbatim}

\begin{enumerate}
\def\labelenumi{\arabic{enumi}.}
\setcounter{enumi}{10}
\tightlist
\item
  What is the final set of explanatory variables that the AIC method
  suggests we use to predict temperature in our multiple regression? How
  much of the observed variance does this model explain? Is this an
  improvement over the model using only depth as the explanatory
  variable?
\end{enumerate}

\begin{quote}
Answer: The final output of the step() suggests that all three variables
of ``Year'', ``Day'', and ``Depth'' are signficiant in determining the
mean recorded temperature of lakes. The final model explains 74.11897\%
of the variability in mean recorded temperature which is slightly higher
than that of the single-variable model. The single variable model does
have a significantly higher AIC (53762) than that of the fuller model
(53674), meaning the fuller model is an improvement upon the
single-variable.
\end{quote}

\begin{center}\rule{0.5\linewidth}{0.5pt}\end{center}

\hypertarget{analysis-of-variance}{%
\subsection{Analysis of Variance}\label{analysis-of-variance}}

\begin{enumerate}
\def\labelenumi{\arabic{enumi}.}
\setcounter{enumi}{11}
\tightlist
\item
  Now we want to see whether the different lakes have, on average,
  different temperatures in the month of July. Run an ANOVA test to
  complete this analysis. (No need to test assumptions of normality or
  similar variances.) Create two sets of models: one expressed as an
  ANOVA models and another expressed as a linear model (as done in our
  lessons).
\end{enumerate}

\begin{Shaded}
\begin{Highlighting}[]
\CommentTok{\#12}
\NormalTok{lake\_anova }\OtherTok{\textless{}{-}} \FunctionTok{aov}\NormalTok{(}\AttributeTok{data =}\NormalTok{ ntl\_july, temperature\_C }\SpecialCharTok{\textasciitilde{}}\NormalTok{ lakename)}
\FunctionTok{print}\NormalTok{(}\FunctionTok{summary}\NormalTok{(lake\_anova))}
\end{Highlighting}
\end{Shaded}

\begin{verbatim}
##               Df Sum Sq Mean Sq F value Pr(>F)    
## lakename       8  21642  2705.2      50 <2e-16 ***
## Residuals   9719 525813    54.1                   
## ---
## Signif. codes:  0 '***' 0.001 '**' 0.01 '*' 0.05 '.' 0.1 ' ' 1
\end{verbatim}

\begin{Shaded}
\begin{Highlighting}[]
\NormalTok{lake\_lm }\OtherTok{\textless{}{-}} \FunctionTok{lm}\NormalTok{(}\AttributeTok{data =}\NormalTok{ ntl\_july, temperature\_C }\SpecialCharTok{\textasciitilde{}}\NormalTok{ lakename)}
\FunctionTok{print}\NormalTok{(}\FunctionTok{summary}\NormalTok{(lake\_lm))}
\end{Highlighting}
\end{Shaded}

\begin{verbatim}
## 
## Call:
## lm(formula = temperature_C ~ lakename, data = ntl_july)
## 
## Residuals:
##     Min      1Q  Median      3Q     Max 
## -10.769  -6.614  -2.679   7.684  23.832 
## 
## Coefficients:
##                          Estimate Std. Error t value Pr(>|t|)    
## (Intercept)               17.6664     0.6501  27.174  < 2e-16 ***
## lakenameCrampton Lake     -2.3145     0.7699  -3.006 0.002653 ** 
## lakenameEast Long Lake    -7.3987     0.6918 -10.695  < 2e-16 ***
## lakenameHummingbird Lake  -6.8931     0.9429  -7.311 2.87e-13 ***
## lakenamePaul Lake         -3.8522     0.6656  -5.788 7.36e-09 ***
## lakenamePeter Lake        -4.3501     0.6645  -6.547 6.17e-11 ***
## lakenameTuesday Lake      -6.5972     0.6769  -9.746  < 2e-16 ***
## lakenameWard Lake         -3.2078     0.9429  -3.402 0.000672 ***
## lakenameWest Long Lake    -6.0878     0.6895  -8.829  < 2e-16 ***
## ---
## Signif. codes:  0 '***' 0.001 '**' 0.01 '*' 0.05 '.' 0.1 ' ' 1
## 
## Residual standard error: 7.355 on 9719 degrees of freedom
## Multiple R-squared:  0.03953,    Adjusted R-squared:  0.03874 
## F-statistic:    50 on 8 and 9719 DF,  p-value: < 2.2e-16
\end{verbatim}

\begin{enumerate}
\def\labelenumi{\arabic{enumi}.}
\setcounter{enumi}{12}
\tightlist
\item
  Is there a significant difference in mean temperature among the lakes?
  Report your findings.
\end{enumerate}

\begin{quote}
Answer: According to both the ANOVA and linear models, a lake's given
name has a significant effect in determining the mean recorded
temperature. The ANOVA outputs a signficiant low p-value, meaning the
null hypthesis (lake name has no significant effect on mean recorded
temperature) is rejected. The linear model compares each lake to the
``intercept'' or a given lake's mean recorded temperature. It outputs
significantly low p-values for each lakename as well, meaning that each
lake has a significantly different mean temperature from the lake used
as intercept.
\end{quote}

\begin{enumerate}
\def\labelenumi{\arabic{enumi}.}
\setcounter{enumi}{13}
\tightlist
\item
  Create a graph that depicts temperature by depth, with a separate
  color for each lake. Add a geom\_smooth (method = ``lm'', se = FALSE)
  for each lake. Make your points 50 \% transparent. Adjust your y axis
  limits to go from 0 to 35 degrees. Clean up your graph to make it
  pretty.
\end{enumerate}

\begin{Shaded}
\begin{Highlighting}[]
\CommentTok{\#14.}

\FunctionTok{ggplot}\NormalTok{(}\AttributeTok{data =}\NormalTok{ ntl\_july, }\FunctionTok{aes}\NormalTok{(}\AttributeTok{y =}\NormalTok{ temperature\_C, }\AttributeTok{x =}\NormalTok{ depth,}
                            \AttributeTok{color =}\NormalTok{ lakename)) }\SpecialCharTok{+} 
  \FunctionTok{geom\_point}\NormalTok{(}\AttributeTok{alpha =} \FloatTok{0.5}\NormalTok{, }\AttributeTok{size =}\NormalTok{ .}\DecValTok{5}\NormalTok{) }\SpecialCharTok{+}
  \FunctionTok{geom\_smooth}\NormalTok{(}\AttributeTok{method =} \StringTok{"lm"}\NormalTok{, }\AttributeTok{se =}\NormalTok{ F, }\AttributeTok{size =} \FloatTok{0.75}\NormalTok{) }\SpecialCharTok{+}
  \FunctionTok{ylim}\NormalTok{(}\FunctionTok{c}\NormalTok{(}\DecValTok{0}\NormalTok{,}\DecValTok{35}\NormalTok{)) }\SpecialCharTok{+} 
  \FunctionTok{labs}\NormalTok{(}\AttributeTok{x =} \StringTok{"Depth(m)"}\NormalTok{, }\AttributeTok{y =} \StringTok{"Temperature(C)"}\NormalTok{, }\AttributeTok{title =} \StringTok{"Mean July Lake Temp.}\SpecialCharTok{\textbackslash{}n}\StringTok{ vs.}
\StringTok{       Depth Level"}\NormalTok{, }\AttributeTok{color =} \StringTok{"Lake"}\NormalTok{)}
\end{Highlighting}
\end{Shaded}

\begin{verbatim}
## `geom_smooth()` using formula 'y ~ x'
\end{verbatim}

\begin{verbatim}
## Warning: Removed 73 rows containing missing values (geom_smooth).
\end{verbatim}

\includegraphics{A06_GLMs_files/figure-latex/scatterplot.2-1.pdf}

\begin{enumerate}
\def\labelenumi{\arabic{enumi}.}
\setcounter{enumi}{14}
\tightlist
\item
  Use the Tukey's HSD test to determine which lakes have different
  means.
\end{enumerate}

\begin{Shaded}
\begin{Highlighting}[]
\CommentTok{\#15}
\FunctionTok{TukeyHSD}\NormalTok{(lake\_anova)}
\end{Highlighting}
\end{Shaded}

\begin{verbatim}
##   Tukey multiple comparisons of means
##     95% family-wise confidence level
## 
## Fit: aov(formula = temperature_C ~ lakename, data = ntl_july)
## 
## $lakename
##                                          diff        lwr        upr     p adj
## Crampton Lake-Central Long Lake    -2.3145195 -4.7031913  0.0741524 0.0661566
## East Long Lake-Central Long Lake   -7.3987410 -9.5449411 -5.2525408 0.0000000
## Hummingbird Lake-Central Long Lake -6.8931304 -9.8184178 -3.9678430 0.0000000
## Paul Lake-Central Long Lake        -3.8521506 -5.9170942 -1.7872070 0.0000003
## Peter Lake-Central Long Lake       -4.3501458 -6.4115874 -2.2887042 0.0000000
## Tuesday Lake-Central Long Lake     -6.5971805 -8.6971605 -4.4972005 0.0000000
## Ward Lake-Central Long Lake        -3.2077856 -6.1330730 -0.2824982 0.0193405
## West Long Lake-Central Long Lake   -6.0877513 -8.2268550 -3.9486475 0.0000000
## East Long Lake-Crampton Lake       -5.0842215 -6.5591700 -3.6092730 0.0000000
## Hummingbird Lake-Crampton Lake     -4.5786109 -7.0538088 -2.1034131 0.0000004
## Paul Lake-Crampton Lake            -1.5376312 -2.8916215 -0.1836408 0.0127491
## Peter Lake-Crampton Lake           -2.0356263 -3.3842699 -0.6869828 0.0000999
## Tuesday Lake-Crampton Lake         -4.2826611 -5.6895065 -2.8758157 0.0000000
## Ward Lake-Crampton Lake            -0.8932661 -3.3684639  1.5819317 0.9714459
## West Long Lake-Crampton Lake       -3.7732318 -5.2378351 -2.3086285 0.0000000
## Hummingbird Lake-East Long Lake     0.5056106 -1.7364925  2.7477137 0.9988050
## Paul Lake-East Long Lake            3.5465903  2.6900206  4.4031601 0.0000000
## Peter Lake-East Long Lake           3.0485952  2.2005025  3.8966879 0.0000000
## Tuesday Lake-East Long Lake         0.8015604 -0.1363286  1.7394495 0.1657485
## Ward Lake-East Long Lake            4.1909554  1.9488523  6.4330585 0.0000002
## West Long Lake-East Long Lake       1.3109897  0.2885003  2.3334791 0.0022805
## Paul Lake-Hummingbird Lake          3.0409798  0.8765299  5.2054296 0.0004495
## Peter Lake-Hummingbird Lake         2.5429846  0.3818755  4.7040937 0.0080666
## Tuesday Lake-Hummingbird Lake       0.2959499 -1.9019508  2.4938505 0.9999752
## Ward Lake-Hummingbird Lake          3.6853448  0.6889874  6.6817022 0.0043297
## West Long Lake-Hummingbird Lake     0.8053791 -1.4299320  3.0406903 0.9717297
## Peter Lake-Paul Lake               -0.4979952 -1.1120620  0.1160717 0.2241586
## Tuesday Lake-Paul Lake             -2.7450299 -3.4781416 -2.0119182 0.0000000
## Ward Lake-Paul Lake                 0.6443651 -1.5200848  2.8088149 0.9916978
## West Long Lake-Paul Lake           -2.2356007 -3.0742314 -1.3969699 0.0000000
## Tuesday Lake-Peter Lake            -2.2470347 -2.9702236 -1.5238458 0.0000000
## Ward Lake-Peter Lake                1.1423602 -1.0187489  3.3034693 0.7827037
## West Long Lake-Peter Lake          -1.7376055 -2.5675759 -0.9076350 0.0000000
## Ward Lake-Tuesday Lake              3.3893950  1.1914943  5.5872956 0.0000609
## West Long Lake-Tuesday Lake         0.5094292 -0.4121051  1.4309636 0.7374387
## West Long Lake-Ward Lake           -2.8799657 -5.1152769 -0.6446546 0.0021080
\end{verbatim}

16.From the findings above, which lakes have the same mean temperature,
statistically speaking, as Peter Lake? Does any lake have a mean
temperature that is statistically distinct from all the other lakes?

\begin{quote}
Answer: Crampton Lake \& Central Long Lake, Ward Lake \& Crampton Lake,
Hummingbird Lake \& East Long Lake, Tuesday Lake \& East Long Lake,
Tuesday Lake \& Hummingbird Lake, West Long Lake \& Hummingbird Lake,
Peter Lake \& Paul Lake, Ward Lake \& Paul Lake, West Long Lake \&
Tuesday Lake are pairs of lake with p-adjacent values above 0.05. Their
mean recorded temperatures are not signficantly different.There is no
one lake with a mean recorded temperature signficantly different from
the others.
\end{quote}

\begin{enumerate}
\def\labelenumi{\arabic{enumi}.}
\setcounter{enumi}{16}
\tightlist
\item
  If we were just looking at Peter Lake and Paul Lake. What's another
  test we might explore to see whether they have distinct mean
  temperatures?
\end{enumerate}

\begin{quote}
Answer: A two sample t test to test whether the mean recorded
temperatures of Peter and Paul Lake are significantly different. The
hypotheseses would be: H0: mean\_temp\_Paul - mean\_temp\_Peter == 0 /
Ha: mean\_temp\_Paul - mean\_temp\_Peter != 0.
\end{quote}

\end{document}
